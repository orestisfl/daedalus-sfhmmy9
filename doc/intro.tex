\section{Περιγραφή της εφαρμογής}
\subsection{Ιδέα \& Υλοποίηση}
\newcommand{\screenshot}[2]{%
\begin{subfigure}{0.5\textwidth}%
\centering%
\includegraphics[keepaspectratio, width=0.4\textwidth]{screenshots/#1}%
\caption{#2}
\label{fig:#1}%
\end{subfigure}%
}
Το CellLock είναι μια εφαρμογή που στοχεύει στην προστασία του κινητού σας από κλοπή ή απώλεια μέ την χρήση μιας εξωτερικής συσκευής και bluetooth.
Η λειτουργία του συνοψίζεται παρακάτω:
\begin{itemize}
\item Μετά από την σύνδεση του κινητού  με το κύκλωμα μέσω bluetooth, δίνεται η δυνατότητα στο χρήστη να μπει στην λειτουργία "protection".

\item Όταν το κινητό βρεθεί σε μεγαλύτερη από την προκαθορισμένη απόσταση από το κύκλωμα (λειτουργία "danger") ενεργοποιείται ένας συναγερμός τόσο στο κινητό, που απαιτεί έναν κωδικό πρόσβασης για να απενεργοποιηθεί, όσο και στο κύκλωμα.
Με αυτό τον τρόπο αν τα κινητό κλαπεί ή ξεχαστεί, ο χρήστης θα το αντιληφθεί αμέσως.
\item Δίνεται η δυνατότητα στον χρήστη να αναζητήσει το κινητό του πατώντας ένα κουμπί πάνω στη συσκευή.
Αυτό, προκαλεί την αναπαραγωγή ενός προεπιλεγμένου ήχου και σκοπεύει στην εύρεση του κινητού αν αυτό έχει χαθεί.
Αντίστοιχη λειτουργία υπάρχει και από την πλευρά του κινητού για την εύρεση της συσκευής.
\end{itemize}

Στόχος μας είναι να υλοποιήσουμε μια εφαρμογή αρκετά απλή για να μπορεί να την χρησιμοποιήσει ο οποιοσδήποτε χρήστης χωρίς τεχνικές γνώσεις.

\renewcommand{\figurename}{Σχήμα}
\begin{figure}[htbp]
        \screenshot{start}{Αρχική οθόνη}%
        \screenshot{locked}{Οθόνη για εισαγωγή κωδικού για ξεκλείδωμα}
        \screenshot{bluetooth-open}{Ερώτηση για άνοιγμα bluetooth}%
        \screenshot{not-connected}{Κατάσταση χωρίς σύνδεση με τα bluetooth του κυκλώματος}
        \screenshot{not-protected}{Κατάσταση χωρίς προστασία}%
        \screenshot{protected}{Κατάσταση με προστασία}
        \screenshot{ring}{Κατάσταση αναπαραγωγής ήχου για την εύρεση της συσκευής}
        \caption{Οι διάφορες καταστάσεις της εφαρμογής}
        \label{fig:screenshots}
\end{figure}

Στο \hyperref[fig:screenshots]{σχήμα \ref{fig:screenshots}}
φαίνονται κάποια χαρακτηριστικά screenshots από την εφαρμογή.

Ο κώδικας της εφαρμογής, τα \LaTeX{} αρχεία για την παραγωγή αυτής της αναφοράς και διάφορα διαγράμματα βρίσκονται στο
\url{https://github.com/orestisf1993/daedalus-sfhmmy9}.

\subsection{Περιγραφή hardware \& κώδικα arduino}
\renewcommand{\figurename}{Σχήμα}
\begin{figure}[htb]
    \centering
    \includegraphics[keepaspectratio, width=\linewidth]{hardware/sketch_bb}
    \caption{Το κύκλωμα που χρησιμοποιήθηκε}
    \label{fig:hardware}
\end{figure}
Από πλευρά hardware χρησιμοποιήθηκαν:
\begin{itemize}
\item 1 \href{https://www.arduino.cc/en/Main/ArduinoBoardUno}{Arduino UNO Rev3}
\item 2 HC06 - Bluetooth Module for Arduino
\item 1 breadboard
\item 3 LEDάκια
\item 1 μπαταρία \SI{9}{\volt} με έναν \href{http://playground.arduino.cc/Learning/9VBatteryAdapter}{9V battery adapter} για arduino.
\item 1 buzzer.
\item 1 κομπί.
\item Καλώδια και αντιστάσεις.
\end{itemize}

Στο arduino εκτελούνται όλες οι βασικές λειτουργίες.
Ο κυρίως κώδικας βρίσκεται στο αρχείο \url{Arduino/daerduino/daerduino.ino}:
\begin{itemize}
\item \sloppy Η συνάρτηση \lstinline[language=C++]!bool emergencyButtonPressed()! εκτελείται σε κάθε επανάληψη της κύριας \lstinline[language=C++]!loop()! και αν ανιχνεύσει συνεχόμενο πάτημα του κουμπιού επί τουλάχιστον 1 δευτερόλεπτο επιστρέφει \lstinline[language=C++]!true!.

\item Η συνάρτηση \lstinline[language=C++]!String readBT()! διαβάζει χαρακτήρες από το bluetooth μέχρι να συναντήσει τον χαρακτήρα \lstinline[language=C++]!';'!.

\item \sloppy Η συνάρτηση \lstinline[language=C++]!void interruptBluetooth()! καλείται σαν timer interrupt
κάθε \lstinline[language=C++]!INTERRUPT_FREQUENCY_MICROS! \si{\micro\second}
και στέλνει στη συνδεδεμένη συσκευή android τον χαρακτήρα \lstinline[language=C++]!"C"! ώστε να επιβεβαιωθεί η σύνδεση και διαβάζει τις εντολές που στέλνει η συσκευή android.
Οι διαθέσιμες εντολές είναι:
\begin{itemize}
\item \lstinline[language=C++]!"RING"!: ενεργοποιεί το alarm: \lstinline[language=C++]!playRing = true;!.
\item \lstinline[language=C++]!"RSTOP"!: απενεργοποιεί το alarm: \lstinline[language=C++]!playRing = false;!.
\item \lstinline[language=C++]!"POFF"!: απενεργοποίηση προστασίας.
\item \lstinline[language=C++]!"PON"!: κατάσταση προστασίας.
\item \lstinline[language=C++]!"SEMI"!: κατάσταση ημι-κινδύνου.
\item \lstinline[language=C++]!"DANG"!: κατάσταση κινδύνου.
\end{itemize}

\item Η συνάρτηση \lstinline[language=C++]!void updateLEDs()! διαχειρίζεται τα LEDs ανάλογα με την τρέχουσα κατάσταση.

\item Η συνάρτηση \lstinline[language=C++]!void updateLEDs()! διαχειρίζεται τα LEDs ανάλογα με την τρέχουσα κατάσταση.

\item Η συνάρτηση \lstinline[language=C++]!void loop()! είναι η κυρίως λούπα της εφαρμογής, αναλαμβάνει:
\begin{itemize}
\item Το reset του watchdog.
\item Την αναπαραγωγή του alarm μέσω της \lstinline[language=C++]!sing()! εάν η \lstinline[language=C++]!alarmShouldPlay()! επιστρέψει \lstinline[language=C++]!true!.
\item Την αποστολή της εντολής \lstinline[language=C++]!"RING"! στην συσκευή android σύμφωνα με την επιστρεφόμενη τιμή της \lstinline[language=C++]!emergencyButtonPressed()!.
\end{itemize}
\end{itemize}

Στο \url{Arduino/daerduino/supermario.ino}
βρίσκεται η συνάρτηση \lstinline[language=C++]!sing()! που αναλαμβάνει την αναπαραγωγή του alarm στο buzzer.

\section{Περιγραφή του android app}
% Backup original figurename command to restore later.
\let\originalfigurename\figurename
% Set figurename command.
\renewcommand{\figurename}{Εικόνα}
\newcommand{\screenshot}[2]{%
\begin{subfigure}{0.5\textwidth}%
\centering%
\includegraphics[keepaspectratio, width=0.4\textwidth]{screenshots/#1}%
\caption{#2}
\label{fig:#1}%
\end{subfigure}%
}

\begin{figure}[htbp]
        \screenshot{start}{Αρχική οθόνη}%
        \screenshot{locked}{Οθόνη για εισαγωγή κωδικού για ξεκλείδωμα}
        \screenshot{bluetooth-open}{Ερώτηση για άνοιγμα bluetooth}%
        \screenshot{not-connected}{Κατάσταση χωρίς σύνδεση με τα bluetooth του κυκλώματος}
        \screenshot{not-protected}{Κατάσταση χωρίς προστασία}%
        \screenshot{protected}{Κατάσταση με προστασία}
        \screenshot{ring}{Κατάσταση αναπαραγωγής ήχου για την εύρεση της συσκευής}
        \caption{Οι διάφορες καταστάσεις της εφαρμογής}
\end{figure}

Για την ανάπτυξη της εφαρμογής μας χρησιμοποιήθηκε το android studio και το αντίστοιχο API για Bluetooth.
Παρακάτω υπάρχουν κάποια επιμέρους ζητήματα που αναλύονται περισσότερο.

\subsection{Ισχύς Ζεύξης Συσκευών}
Βασική απαίτηση της εφαρμογής μας ήταν να έχει την δυνατότητα εκτίμησης της απόστασης.
Το API του Android όμως δεν δίνει την τιμή RSSI (Received Signal Strength Indication) όταν είναι συνδεδεμένο ένα κινητό με ένα bluetooth.
Η τιμή του rssi δίνεται μόνο κατά την διαδικασία της αναζήτησης συσκευών (discovery) κατά την οποία όμως δεν μπορούμε να βρούμε συσκευές με τις οποίες ήμαστε ήδη συνδεδεμένοι.
Συνεπώς, χρησιμοποιήθηκαν 2 bluetooth στην εφαρμογή μας: ένα για discovery και ένα για connection.

Για να μπορούμε να μετρήσουμε την απόσταση των δύο συσκευών μας, είναι
απαραίτητο να γνωρίζουμε την ισχύ της ζεύξης των δύο συσκευών.
Το API του android μας επιστρέφει την τιμή του rssi σε dB με τον τρόπο που αναφέρθηκε.

Μέσα από διάφορα πειράματα που εκτελέστηκαν κατά την διάρκεια του διαγωνισμού
φτάσαμε στα παρακάτω συμπεράσματα:
\begin{itemize}
\item Η απόσταση στην οποία πρέπει να ενεργοποιείται ο κίνδυνος είναι 5 με 6
μέτρα.
\item Σε απόσταση 4 με 5 μέτρα πρέπει να βρισκόμαστε στην κατάσταση στην οποία
υπάρχει προειδοποίηση για πιθανό κίνδυνο.
\end{itemize}

Οπότε, καταλήξαμε πως για τιμές του rssi μεγαλύτερες από -75dB βρισκόμαστε σε
ασφαλή κατάσταση.
Για τιμές μεταξύ -75dB και -80dB βρισκόμαστε σε κατάσταση πιθανού
κινδύνου, ενώ σε τιμές μικρότερες από -80dB σε κατάσταση κινδύνου.

\subsection{Λογική του Προγράμματος}
Επειδή η εφαρμογή μας πρέπει να λειτουργεί με κάποιου είδους μνήμη στις εισόδους που δέχεται (πάτημα κουμπιών, ισχύς σήματος) χρησιμοποιήθηκε η λογική των states για να δομηθεί το πρόγραμμά μας.
Τα states καθώς και οι μεταβάσεις φαίνονται στο
\renewcommand{\figurename}{Σχήμα}
\hyperref[fig:stdiag]{\figurename{} \ref{fig:stdiag}}.
\begin{figure}[H]
    \centering
    \includegraphics[keepaspectratio, width=\linewidth]{images/state-diagram}
    \caption{Το διάγραμμα καταστάσεων}
    \label{fig:stdiag}
\end{figure}

\subsection{Ομάδες \& χρονοδιάγραμμα}
Κατά τη διάρκεια του διαγωνισμού χωριστήκαμε στις εξής ομάδες:
\begin{itemize}
\item Χαμζας – Παπουδάκης (android – bluetooth API development).
\item Κίρτσιος (android UI develepment).
\item Τσιριγώτης (distance measure modeling).
\item Φλώρος (Software-Hardware interface).
\end{itemize}

Στόχος της πρώτης μέρας ήταν να ολοκληρωθεί τόσο το hardware όσο και το software και να καταφέρουμε να πετύχουμε πλήρη επικοινωνία μεταξύ πομποδέκτη και κινητού.
Στη συνέχεια, στην δεύτερη μέρα σκοπός ήταν να εκτιμηθεί η απόσταση ανάμεσα στις δύο συσκευές μέσω της ισχύος σήματος ζεύξης και να ελενχθεί πλήρως η εφαρμογή.


