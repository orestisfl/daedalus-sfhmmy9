\section{Προβλήματα}
Το βασικό πρόβλημα που αντιμετωπίσαμε ήταν πως το API του Android δεν δίνει την τιμή RSSI (Received Signal Strength Indication) όταν είναι συνδεδεμένο ένα κινητό με ένα bluetooth.
Η τιμή του rssi δίνεται μόνο κατά την διαδικασία της αναζήτησης συσκευών (discovery) κατά την οποία όμως δεν μπορούμε να βρούμε συσκευές με τις οποίες ήμαστε ήδη συνδεδεμένοι.
Οπότε, αποφασίσαμε να χρησιμοποιήσουμε 2 και όχι 1 bluetooth module στο hardware: ένα για discovery και ένα για connection.

Ένα άλλο πρόβλημα που άλλαξε τον αρχικό σχεδιασμό μας ήταν πως μερικές φορές τα μηνύματα που ανταλλάσσονταν μεταξύ της συσκευής και το κινητού χάνονταν και προκαλούνταν αποσυγχρονισμός στην λειτουργία τους.
Για να το αντιμετωπίσουμε αυτό δημιουργήσαμε 1 επιπλέον thread από τη μεριά του κινητού που επαναλαμβανόμενα έστελνε το state του κινητού στο arduino.
