%~ xelatex -ini -shell-escape -job-name="preamble" "&xelatex preamble.tex\dump"
\RequirePackage[l2tabu, orthodox]{nag}
% Specify the type of document
\documentclass[a4paper, titlepage]{article}
%\documentclass[draft]{report}

% packages
% https://foss.ntua.gr/wiki/index.php/%CE%95%CE%BB%CE%BB%CE%B7%CE%BD%CE%B9%CE%BA%CE%B1_%CF%83%CF%84%CE%BF_TeX/LaTeX_%CE%BC%CE%B5_%CF%84%CE%BF_XeTeX#.CE.97yphenation
% title in pic:
\usepackage{titlepic}
\usepackage{tikz}
\usepackage[margin=1in]{geometry}
\usepackage{fancyhdr}
\pagestyle{fancy}
\fancyhf{}
\renewcommand{\headrulewidth}{0pt}
\fancyhead[R]{\thepage}
\usepackage{subcaption}
% Select alternative section titles
% https://www.ctan.org/pkg/titlesec
\usepackage{titlesec}

% For graphics.
% http://www.kwasan.kyoto-u.ac.jp/solarb6/usinggraphicx.pdf
% https://ctan.org/pkg/graphicx
\usepackage{graphicx}

% provides LaTeX the ability to create hyperlinks within the document.
% https://www.ctan.org/pkg/hyperref
% http://en.wikibooks.org/wiki/LaTeX/Hyperlinks#Hyperref
\usepackage[bookmarks=true,colorlinks=true,linkcolor=blue]{hyperref}
% for going to the top of an image when a figure reference is clicked: http://stackoverflow.com/a/21251925/3430986%
\usepackage{caption}
\captionsetup[lstlisting]{box=colorbox,boxcolor=gray,font={color=white},labelsep=endash,skip=2pt}
% http://tex.stackexchange.com/questions/118713/is-microtype-fully-supported-now-by-xelatex-if-not-how-can-i-keep-myself-up-to
% http://tug.org/pipermail/xetex/2013-April/024263.html
\usepackage{amsmath}
\usepackage{amsfonts}
\usepackage{siunitx}
\sisetup{locale=FR}
\usepackage{bm}
\usepackage{fancyref}
%\usepackage{varwidth}

% used for figures in multicol enviroment
%\usepackage{multicol}
%\newenvironment{Figure}
%  {\par\medskip\noindent\minipage{\linewidth}}
%  {\endminipage\par\medskip}

% http://ctan.org/pkg/enumerate
\usepackage{enumerate}
% http://tex.stackexchange.com/a/56953/78791
\newcommand{\argmin}{\operatornamewithlimits{argmin}}
% http://tex.stackexchange.com/a/107190/78791
\newcommand{\norm}[1]{\left\lVert#1\right\rVert}
\newcommand{\abs}[1]{\left\lvert#1\right\rvert}
\newcommand{\overbar}[1]{\mkern 1.5mu\overline{\mkern-1.5mu#1\mkern-1.5mu}\mkern 1.5mu}
% bad fonts for implies?
\renewcommand{\implies}{=\!\Rightarrow}
\usepackage{extarrows}

% Used for pgf plots. Taken from https://github.com/matlab2tikz/matlab2tikz
% also see http://www.howtotex.com/packages/beautiful-matlab-figures-in-latex/
%\usepackage{pgfplots}
%\pgfplotsset{compat=newest}
%\pgfplotsset{plot coordinates/math parser=false}
%\newlength\figureheight
%\newlength\figurewidth
\renewcommand*{\ttdefault}{Hack}
% for code with higlighting
\usepackage{listings}
\usepackage{lstautogobble}
% Change the caption name for the listings
\renewcommand{\lstlistingname}{Code}
\usepackage{color}
\definecolor{mygreen}{rgb}{0,0.6,0}
\definecolor{mygray}{rgb}{0.5,0.5,0.5}
\definecolor{mymauve}{rgb}{0.58,0,0.82}
\lstset{ %
  backgroundcolor=\color{white},   % choose the background color; you must add \usepackage{color} or \usepackage{xcolor}
  basicstyle=\normalsize\ttfamily, % the size of the fonts that are used for the code
  breakatwhitespace=false,         % sets if automatic breaks should only happen at whitespace
  breaklines=false,                % sets automatic line breaking
  captionpos=t,                    % sets the caption-position to bottom
  commentstyle=\color{mygreen},    % comment style
  escapeinside={\%*}{*)},          % if you want to add LaTeX within your code
  extendedchars=true,              % lets you use non-ASCII characters; for 8-bits encodings only, does not work with UTF-8
  frame=single,                    % adds a frame around the code
  keepspaces=true,                 % keeps spaces in text, useful for keeping indentation of code (possibly needs columns=flexible)
  keywordstyle=\color{blue},       % keyword style
  language=Java,                    % the language of the code
  numbers=left,                    % where to put the line-numbers; possible values are (none, left, right)
  numbersep=5pt,                   % how far the line-numbers are from the code
  numberstyle=\tiny\color{mygray}, % the style that is used for the line-numbers
  rulecolor=\color{black},         % if not set, the frame-color may be changed on line-breaks within not-black text (e.g. comments (green here))
  showspaces=false,                % show spaces everywhere adding particular underscores; it overrides 'showstringspaces'
  showstringspaces=false,          % underline spaces within strings only
  showtabs=false,                  % show tabs within strings adding particular underscores
  stepnumber=1,                    % the step between two line-numbers. If it's 1, each line will be numbered
  stringstyle=\color{mymauve},     % string literal style
  tabsize=4,                       % sets default tabsize to 2 spaces
  title=\lstname                   % show the filename of files included with \lstinputlisting; also try caption instead of title
}
\lstdefinestyle{Bash}
{language=bash,
alsoletter={:~$},
morekeywords=[2]{peter@kbpet:},
keywordstyle=[2]{\color{red}},
literate={\$}{{\textcolor{red}{\$}}}1
         {:}{{\textcolor{red}{:}}}1
         {~}{{\textcolor{red}{\textasciitilde}}}1,
}

\usepackage{multirow}
\usepackage{placeins}
\usepackage{etoolbox}
\usepackage{booktabs}
% no newpage at chapter
\makeatletter
\patchcmd{\chapter}{\if@openright\cleardoublepage\else\clearpage\fi}{}{}{}
\makeatother
\setcounter{tocdepth}{3} % subsub has numbers
\setcounter{secnumdepth}{3} % subsub in toc

\usepackage{adjustbox}  % for maxsizebox: http://stackoverflow.com/a/29143366/3430986
\usepackage{float}  % for figure H.

\usepackage[official]{eurosym} % Euros.
% http://tex.stackexchange.com/a/110979/78791
\DeclareRobustCommand{\officialeuro}{%
  \ifmmode\expandafter\text\fi
  {\fontencoding{U}\fontfamily{eurosym}\selectfont e}}